\documentclass{article} % For LaTeX2e
\usepackage{iclr2018_conference,times}
\usepackage{hyperref}
\usepackage{url}
\usepackage{amssymb}


%%% custom packages by JPG and LA:
\usepackage[most]{tcolorbox}
\usepackage{amsmath}

\usepackage{upgreek}

\usepackage{soul}
\usepackage{algorithm}
\usepackage{algpseudocode}

\newcommand{\vect}[1]{\vec{#1}}
\DeclareMathOperator{\Feat2Vec}{Feat2Vec}
\DeclareMathOperator{\Word2Vec}{Word2Vec}
\DeclareMathOperator{\MN}{MN}
\newcommand{\dotp}{\boldsymbol{\cdot} }
\renewcommand{\cite}[1]{\citep{#1}}
\usepackage{amsthm}
\usepackage{thmtools,thm-restate}
\declaretheorem{theorem}
\title{Feat2Vec:  Semantic Modularity for Continuous Representations of Data with Features}

% Authors must not appear in the submitted version. They should be hidden
% as long as the \iclrfinalcopy macro remains commented out below.
% Non-anonymous submissions will be rejected without review.

\author{Antiquus S.~Hippocampus, Natalia Cerebro \& Amelie P. Amygdale \thanks{ Use footnote for providing further information
about author (webpage, alternative address)---\emph{not} for acknowledging
funding agencies.  Funding acknowledgements go at the end of the paper.} \\
Department of Computer Science\\
Cranberry-Lemon University\\
Pittsburgh, PA 15213, USA \\
\texttt{\{hippo,brain,jen\}@cs.cranberry-lemon.edu} \\
\And
Ji Q. Ren \& Yevgeny LeNet \\
Department of Computational Neuroscience \\
University of the Witwatersrand \\
Joburg, South Africa \\
\texttt{\{robot,net\}@wits.ac.za} \\
\AND
Coauthor \\
Affiliation \\
Address \\
\texttt{email}
}

% The \author macro works with any number of authors. There are two commands
% used to separate the names and addresses of multiple authors: \And and \AND.
%
% Using \And between authors leaves it to \LaTeX{} to determine where to break
% the lines. Using \AND forces a linebreak at that point. So, if \LaTeX{}
% puts 3 of 4 authors names on the first line, and the last on the second
% line, try using \AND instead of \And before the third author name.

\newcommand{\fix}{\marginpar{FIX}}
\newcommand{\new}{\marginpar{NEW}}

%\iclrfinalcopy % Uncomment for camera-ready version, but NOT for submission.

\begin{document}


\maketitle

\begin{abstract}
Methods that calculate dense embeddings for features in unstructured data---such as words in a document---have proven to be very successful for knowledge representation.
Surprisingly, very little work has focused on methods for structured datasets where there is more than one feature type---this is, datasets that have arbitrary features beyond words.

We study how to estimate  continuous representations for multiple feature types within a dataset,  where each feature type exists in a different higher-dimensional space.
$\Feat2Vec$ is a novel method that calculates embeddings for data with multiple feature types enforcing \textit{semantic modularity} across features---all different feature types exist in a common space.
We demonstrate our work on two datasets.
The first one is collected from a leading educational technology firm, and thus we are able to discover a common continuous space for entities such as universities, courses, textbooks and students.
The second dataset is the public IMDB dataset, and we discover embeddings for entities that include actors, movies and producers.
Our experiments suggest that $\Feat2Vec$ significantly outperforms existing algorithms that do not leverage the structure of the data.

\end{abstract}

\section{Preliminaries}
Informally, in machine learning an \textit{embedding}  of a vector   $\vect{x} \in \mathbb{R}^n$  is another vector $\vect{y} \in \mathbb{R}^r$ that has much lower dimensionality ($r \ll n$) than the original representation, and can be used to replace the original vector  in downstream prediction tasks.
%Embeddings are often used to compress sparse categorical variables, and they result into continuous representation of them.
Embeddings  have multiple advantages, as they enable more efficient training~\cite{mikolov2013distributed},
and unsupervised learning~\cite{schnabel2015evaluation}.
For example, when applied to text,   semantically similar words are mapped to nearby points.

$\Word2Vec$~\cite{mikolov2013efficient,mikolov2013distributed} is an extremely successful software package that contains two embedding functions with the same domain and codomain:
\begin{align}
\Word2Vec :\, &  \bigl\{ \; \vect{x} \in \mathbb{R}^n \mapsto  \vect{y} \in \mathbb{R}^r \; \bigr \}
\intertext{$\Word2Vec$ is suited for calculating embeddings for datasets that consist of documents of words with a vocabulary size of $n$.
Here, $\vect{x}$ is sparse because words (and categorical variables, in general) are  modeled using one-hot encoding.
In this paper we study how to generalize embedding functions for arbitrary datasets with multiple feature types by leveraging their internal structure.
We call this approach  $\Feat2Vec$:}
\Feat2Vec :\, & \bigl\{\; \vect{x\vphantom{\mathcal{f}}} \in \vect{\mathcal{F}} \mapsto \vect{y\vphantom{\mathcal{F}}} \in \mathbb{R}^r \; \bigr\}\\
\vect{ \mathcal{F} } =\,& \bigl[\; \mathbb{R}^{d_1}, \mathbb{R}^{d_2}, \dots, \mathbb{R}^{d_n} \;\bigr]
\end{align}
Here, $\vect{\mathcal{F}}$ is a structured feature set---where each dimension represents a different feature type.
Notice  that  is possible to use   $\Word2Vec$ on structured data by simply flattening the input.
However,   $\Feat2Vec$  leverage the features' structure  to enforce  \textit{semantic modularity} across features.
That is, the embedding of an observation is additive in respect to the embeddings of its feature types;
thus, each feature type  $\mathcal{F}_i$ must be  projected into the same embedding space.
Semantic modularity is useful because it enables making principled comparisons between different feature types.
For example, on an educational dataset that contains many feature types, including  students and courses, semantic modularity enables inferring course preferences for students.
%On the other hand, $\Feat2Vec$ is able to generalize on  $\Word2Vec$ when there is only one feature type $\mathcal{F}(n)$.




\section{Feat2Vec}
The rest of this section describes the three parts of a $\Feat2Vec$ implementation.
\S~\ref{sec:model} describes a  model that infers an  embedding from an observation.
To learn this model we need positive and negative examples:
 a positive example is ``semantic'' and it is observed during training, but negative examples are not.
This is similar how a positive example in $\Word2Vec$ are grammatical co-ocurrences, and negative examples are generated.
\S~\ref{sec:sampling} describes our novel sampling strategy.
%This is similar of how $\Word2Vec$ learns the relationships of words  that co-occur.
%Both $\Word2Vec$ and $\Feat2Vec$ resort to sampling implicit samples,  though we propose a different algorithm.
Finally, in \S~\ref{sec:learning} we describe how to learn a $\Feat2Vec$ model from data.


\subsection{Structured Deep-In Factorization Machine}
\label{sec:model}

\citet{levy2014neural} showed that  a $\Word2Vec$ model can be formalized as a Shallow Factorization Machine~\cite{rendle2010factorization} with two features types: a word  and its context.
This factorization model is a binary classifier that scores the likelihood of an observation $\vect{x} \in \mathbb{R}^n$ being labeled  $y \in \{0,1\}$, as proportional to the sum of the factorized pairwise interactions:

\begin{align}
%\hat y(\vect{x}; \vect{b}, \vect{\upphi}, \vect{G})=
p\bigl(Y=1| \vect{\vphantom{\phi}x}  \bigr ) &=
\sigma \biggl(
                       \sum_{i =1}^{n} \sum_{j=i}^{n}  \; \langle \,  \vect{\beta}_i x_i,\,   \vect{\beta}_j x_j  \, \rangle  %{x}_i {x}_j
  \biggr)
\end{align}
Here, $Y$ is the random variable that defines whether the observation is ``semantic'' (which in practice means whether it occurs in the training data),
 $\vect{\beta}_i$ is a rank-$r$ vector of factors, and $\sigma$ is a sigmoid:
\begin{equation}
\sigma(x) = \log \left(\frac{\exp(x)}{\exp(x+1)} \right)
\end{equation}

However, features  may have some structure that is known beforehand.
For example, consider multiple discrete entity types, like universities, students, and books.
To model this data, a Shallow Factorization Machine would  ignore the structure and simply concatenate the one-hot encoding representation of each entity into a single feature vector.
Feat2Vec relies on an novel extension to factorization called  Structured Deep-In Factorization Machine, which we describe in detail in a \hl{companion paper} \cite{deepfm}.
%On the other hand, Structured Deep-In Factorization Machine considers the structure explicitly.
While Shallow Factorization Machine  learns an embedding per feature, the Structured Deep-In model allows greater flexibility.
For example, consider using images or text on these factorization models.
The shallow model learns an embedding for each word, or an embedding per pixel.
The structured model enables higher-level of abstraction and flexibility, and it can learn an embedding per passage of text, or an embedding per image.


Structured Deep-In Factorization Machine inputs $\vect{\upkappa}$  as a vector of vectors that define  the structure of the groups of features.
Each entry $i$  is a vector of size $d_i$ and it is used to represent the feature types of $\mathcal{F}_i$.
%LA: i'm having trouble following this explanation.
In a shallow model, a feature interacts with all other features, but in the structured model they only interact with features from different groups.
For example, consider a model with four features.
If we define
$
\vect{\upkappa} = \bigl[  \, [1, 2 ], [ 3 , 4 ] \, \bigr]
$,
the structure of the model would not allow $x_1$ to interact with $x_2$, or $x_3$ to interact with $x_4$.
Additionally, for each group of features, the model allows applying a $d_i  \times r $ feature extraction function $\phi_i$.
More formally, this is:
\begin{align}
%\hat y(\vect{x}; \vect{b}, \vect{\upphi}, \vect{G})=
p \bigl(Y=1| \vect{\vphantom{\phi}x};  \vect{\phi}, \vect{\vphantom{\phi}\upkappa} \bigr ) &=
\sigma \biggl(s(\vec{x},\vec{\phi})\biggl)=
\sigma \biggl(
                       \sum_{i =1}^{|{\vect{\upkappa}}|} \sum_{j=i}^{|{\vect{\upkappa}}|}   \bigl\langle \,  \phi_i( \vect{x}_{\vect{\upkappa}_i}), \; \phi_j( \vect{x}_{\vect{\upkappa}_j}) \, \bigr\rangle
  \biggr)
\label{eq:deepfm}
\end{align}

In this notation, $\vect{x}_{\vect{\upkappa}_i}$ is a subvector that contains all of the features that belong to the group $\upkappa_i$, and $s(\vec{x},\vec{\phi})$ is the score function.
Thus, $\vect{x}_{\vect{\upkappa}_i} =  \bigl[ \, x_{i_a} | a \in \{1, 2, \dots, {|\upkappa_i|} \} \, \bigr]$.
The simplest   implementation for $\phi_i$ is a linear fully-connected layer, where the output of the $r$-th entry is:
\[
\phi_i\bigl(\vect{x\vphantom{\beta}}_i; \vect{\beta}\bigr)_r = \sum_{a=1}^{d_i}  \beta_{r_a} x_{i_a}
\]
We build the $\Feat2Vec$ embedding for an observation $\vect{x}$ as:
\begin{equation}
\Feat2Vec(x) =  \bigl [  \;   \phi_q( \vect{x}_{\vect{\upkappa}_1}), \,     \phi_1( \vect{x}_{\vect{\upkappa}_2}), \, \dots, \,  \phi_n( \vect{x}_{\vect{\upkappa}_n})   \; \bigr ]
\end{equation}





\subsection{Sampling}
\label{sec:sampling}

The training dataset for a $\Feat2Vec$ model consists of only semantic observations.
In natural language, these would be documents written by humans.
Since Structured Deep-In Factorization Machine~(Equation \ref{eq:deepfm}) requires positive and negative examples, we also need to supply  observations that are not semantic.
Consider a  feature type $\mathcal{F}_i \in \mathbb{R}^{d_i}$, that exists in very high dimensional space ($d_i$ is large).
For example, this could happen because we are modeling with one-hot encoding a categorical variable with   large number of possible values.
In such scenario, it is overwhelmingly costly to feed the model all negative labels, particularly if the model is fairly sparse.

A shortcut around this is a concept known as \textit{implicit sampling}, where instead of using all of the possible negative labels, one simply samples a fixed number ($k$) from the set of possible negative labels for each positively labelled record.
$\Word2Vec$ makes use of an algorithm called Negative Sampling~\cite{samplingnotes}.
In short, their approach samples a negative observation $w$  from a noise distribution $q$: % $\MN(\{w\},\alpha)$:
\begin{align}
\text{Draw } w  & \thicksim  p(W|Y=0, X)  \\
%                        & \thicksim  \MN(\{w\},\alpha)   \text{\hl{ What is \{w\}?! }}    \\
                        %& \propto \hat p_w^\alpha,\quad  \alpha \in [0,1]
                          &= q(\{w_1, w_2, \dots, w_n \}, \alpha)
\end{align}

Here  $q$ is the empirical frequency of  $w$ in the dataset, exponentiated by $\alpha$, a flattening hyper-parameter:
\begin{equation}
q(W,  \alpha)  =  \frac{ c_X(w)^\alpha }{ \sum\limits_{w' \in W} c_X(w') ^\alpha  }
\end{equation}
%P(wi)=f(wi)3/4?nj=0(f(wj)3/4)

Here, $c_X(w)$ is the number of times a feature $w$ appeared in the training dataset X.
With $\alpha=1$, the noise distribution corresponds to the empirical frequency distribution, and with $\alpha=0$ to the uniform distribution.
Intermediate values are intended to be continuous progressions to these two extremes.

Our main contribution is introducing a new implicit sampling method that enables learning embedding for structured feature sets.
We can learn the correlation of features within a dataset by imputing negative labels, simply by generating ``unobserved'' records as our negative sample.
% replaced n <- p;  because we use p for probabilities
Unlike $\Word2Vec$, we do not constraint features types to be words.
Features types can be  individual numeric columns,  but they need not to be.
By defining feature types as groups of subfeatures (using $\upkappa$ parameter in Equation \ref{eq:deepfm}), the model can reason on more complex entities.
For example, in our experiments on a movie dataset, we use a ``genre'' feature type, where we  group non-mutually exclusive indicators for comedy, action, and drama films under one umbrella.
For more involved feature types (e.g., an image),  one would need to  define a function $\phi$ that builds intermediate layers  to map the entity to an embedding .

We start with a dataset $S^+ $ of records with $p$ feature types.
We then mark all observed records in the training set as positive examples.
For each positive record, we generate $k$ negative labels using the following 2-step algorithm:

\begin{algorithm}
 \caption{Implicit sampling algorithm for $\Feat2Vec$}
\begin{algorithmic}[1]
\Function{Feat2Vec\_Sample}{$S^+, k, \alpha_1, \alpha_2$}
\State $S^-  \leftarrow \emptyset $
\For {$s \in S^+ $}
	\State Draw a random feature type $\upkappa_i  \sim q_1(\{d_i\}_{i=1}^p,\alpha_1) $
	\For {$j \in \{1,\ldots,k\}$}
			\State $\vect{e}  \leftarrow \vect{x}^{(s)}$ \Comment  set initially to be equal to a positive sample
			\State Draw a random (sub)feature  $\vect{r} \sim q_2(\mathrm{X}_{ \upkappa_i },\alpha_2)$
			\State $\vect{e}_{\upkappa_i}  \leftarrow \vect{r}$ \Comment{substitute the $i$-th feature type with a sampled one}
			\State  $S^-  \leftarrow S^-  + \{  \vect{e} \}$
	\EndFor
\EndFor
\State \Return $S^-$
\EndFunction
\end{algorithmic}
\end{algorithm}

Here $q_2(\mathrm{X}_{\upkappa_i}, \alpha_2)$ is the ``flattened'' empirical multionomial distribution over the frequency of the feature values of feature $\upkappa_i$, as in the negative sampling procedure used in $\Word2Vec$ :
\[x_{j}\sim q_2(\mathrm{X}_{\upkappa_i}, \alpha_2),\text{ where } P_{q_2}(x_j=x|\mathrm{X}_{\upkappa_i},\alpha_2)  \
= \frac{c_X(x)^{\alpha_2}}{\sum_{x' \in X_{\upkappa_i}} c_X(x')^{\alpha_2}},\alpha \in [0,1]
\]
%LA: Jose, should we use a different symbol since this is technically a different feature
Where $P_q(x)$ denotes the probability of $x$ under a distribution $q$.
Analogously , $q_1(\{d_i\}_{i=1}^p,\alpha_1)$ draws a feature to negatively sample from a multinomial distribution with probabilitites proportional a feature's complexity, $d_i$. Here, by complexity, we mean the number of parameters the algorithm learns associated with a feature group:
\[
P_{q_1}(\upkappa_i|\{d_i\}_{i=1}^p,\alpha_1)  \
= \frac{d_i^{\alpha_1}}{\sum_{k=1}^p d_k^{\alpha_1}}
\]
$d_i$ may be the dimensionality of a feature, but might also capture other parameters used for learning embeddings, where applicable. For example, if we have multiple intermediate layers to process a set of pixels embeddings into an ``image embedding'' (such as deep convolutional layers), these parameters would count towards $d_i$.

%we should make note at some point of the impossible interpretability of ``intermediately'' populated features/values because of the wierd hyperbolic shape
%of the probability over alpha (not a mix of alpha=0 to alpha=1). See plot I shared with Jose.
Explained in words, our negative sampling method is to randomly select one of the features $\upkappa$ from the noise distribution $q_1(.)$, and, conditional on the feature to be resampled, choose another value $x_j$ for feature $\upkappa$ from a noise distribution $q_2(.)$.
In our application, we use the same class of noise distributions (flattened multinomial) for both levels of sampling, but this need not necessarily be the case.

%jose, resolve the differences here!
This method will sometimes by chance generate negatively labeled samples that \textit{do} exist in our sample of observed records.
For this reason, among others, we employ a technique known as noise contrastive estimation \cite{nce} which uses basic probability laws to adjust the structural statistical model $p(Y=1|\vec{\phi},\vec{x})$ to account for the possibility of random negative labels that appear identical to positively labeled data.
An additional burden of this method, however, is we need to learn additional nuisance parameters $Z_{\vec{x}}$ for each unique record type $\vec{x}$ that transform the score function $s(.)$ into a well-behaved probability distribution that integrates to 1.
This introduces an astronomical amount of new parameters and greatly increase the complexity of the model.
Instead of estimating these, we appeal to the work of \cite{fastnnlang},
who show in the context of language models that setting the $Z_{\vec{x}}=1$ in advance effectively does not change the performance of the model.
\footnote{\cite{fastnnlang} suggests one can set the normalizing constant to 1 because the neural net model has enough free parameters that it will effectively just learn the probabilities itself so that it does not over/under predict the probabilities on average (since that will result in penalties on the loss function). Note that while we follow the same procedure, one could set $Z_{\vec{x}}$ to any positive value in advance and the same logic would follow, if one was worried about astronomically low probabilities.}
Written explicitly, the new structural probability model is:
\[\tilde p(Y=1|\vec{\phi},\vec{x_i}) = \frac{\exp(s(\vec{x}_i,\vec{\phi}))}{\exp(s(\vec{x}_i,\vec{\phi})) + P_{q}(\vec{x}_i| \alpha_1,\alpha_2)}\]
where $P_q(.)$ denotes the unconditional probability of a a record $\vec{x_i}$ being drawn from our negative sampling algorithm, and  $s(x,\vec{\phi})$ is the score function for Factorization Machines, in the syntax of \cite{nce}.
We can show, with relatively little effort,that our 2-step sampler has some interesting theoretical properties. Namely, the embeddings learned under this model will be equivalent to a convex combination of the embeddings learned from $p$ individual Factorization Machines.


\subsection{Learning From data}
\label{sec:learning}

We optimize Noise Contrastive Estimation\cite{nce} loss, which uses basic probability laws to adjust the structural statistical model $p(Y=1|\vec{\phi},\vec{x})$ to account for the possibility of randomly generated negative labels that appear identical to positively labeled data. Written explicitly, the new structural probability model is just:

\begin{equation}
\tilde p(Y=1|\vec{\phi},\vec{x_i}) = \frac{\exp(s(\vec{x}_i,\vec{\phi}))}{\exp(s(\vec{x}_i,\vec{\phi})) + P_{q}(\vec{x}_i| \alpha_1,\alpha_2)}\times Z_{\vec{x}}
\end{equation}
where $P_q(.)$ denotes the unconditional probability of a a record $\vec{x_i}$ being drawn from our implicit sampling algorithm,  $s(x,\vec{\phi})$ is the score function for Factorization Machines, in the syntax of \cite{nce}, and $Z_{\vec{x}_i}$ is a positive learned parameter to guarantee that the probability distribution integrates to 1:
\[
\sum_{\vec{x}_i \in \mathrm{X}} Z_{\vec{x}_i}\times \frac{\exp(s(\vec{x}_i,\vec{\phi}))}{\exp(s(\vec{x}_i,\vec{\phi})) +  P_{q}(\vec{x}_i| \alpha_1,\alpha_2)} = 1
\]
This additional parameter for each unique record,$Z_{\vec{x}_i}$, introduces an astronomical amount of new parameters and greatly increase the complexity of the model. Instead of estimating these, we appeal to the work of \cite{fastnnlang}, who show in the context of language models that setting the $Z_{\vec{x}}=1$ in advance effectively does not change the performance of the model.\footnote{\cite{fastnnlang} suggests one can set the normalizing constant to 1 because the neural net model has enough free parameters that it will effectively just learn the probabilities itself so that it does not over/under predict the probabilities on average (since that will result in penalties on the loss function). Note that while we follow the same procedure, one could set $Z_{\vec{x}}$ to any positive value in advance and the same logic would follow, if one was worried about astronomically low probabilities.}


We can show, with relatively little effort,that our 2-step sampler has some interesting theoretical properties. Namely, the embeddings learned under this model will be equivalent to a convex combination of the embeddings learned from $p$ individual Factorization Machines that also use an NCE loss:

\begin{restatable}{theorem}{convexFM}
\label{thm:thm}
The embeddings learned with $\Feat2Vec$ are  a convex combination of the embeddings learned from $p$ targeted Factorization Machines for each feature in the data.
\end{restatable}

See the proof in the Appendix.


\section{Limitations}

\hl{
Continuous features  require a sensible feature function, and future work could focus on them.

Feature engineering (grouping) is important.
}


\section{Empirical Results}


\section{Relation to Prior Work}


\section{Conclusion}


\appendix
\section{Appendixes}
\subsection{Proof to Theorem \ref{thm:thm}}
\convexFM*
\begin{proof}
Let $S^+_f$ denote the positively labeled records whose corresponding negative samples resample feature $f$. We can express the loss function $L(.)$, the binary cross-entropy of the data given the $\Feat2Vec$ model, as follows:

\begin{align*}
 L(\vec{\mathbf{x}} | \vec{\phi} )  =&   \frac{1}{|S^+|} \sum_{i \in S^+} \Big(\log(\tilde p(Y=1|\vec{\phi},\vec{\mathbf{x}}_i  ) ) + \sum_{\vec{\mathbf{x}}_j \sim q(.|\vec{\mathbf{x}}_i)}^k \log(\tilde p(Y=0|\vec{\phi},\vec{\mathbf{x}}_j)) \Big) \\
  =&  \frac{1}{|S^+|} \sum_{i \in S^+} \Big( \log(\tilde p(Y=1|\vec{\phi},\vec{\mathbf{x}}_i, i \in S_f )p(i \in S^+_f) ) + \sum_{\vec{\mathbf{x}}_{j} \sim q(.|\vec{\mathbf{x}}_i)}^k \log(\tilde p(Y=0|\vec{\phi},\vec{\mathbf{x}}_j,i \in S_f)p(i \in S^+_f)) \Big) \\
  =&  \frac{1}{|S^+|}\sum_{f=1}^p \sum_{i \in S^+_f} \Big( \log(\frac{e^{s(\vec{\mathbf{x}}_i,\vec{\phi}) } p(i \in S^+_f)}{e^{s(\vec{\mathbf{x}}_i,\vec{\phi}) } + P_q(\vec{\mathbf{x}}_i |\vec{\mathbf{x}}_i, i \in S^+_f) }) + \sum_{\vec{\mathbf{x}}_{j} \sim q(.|\vec{\mathbf{x}}_i,i\in S^+_f)}^k \log(\frac{ P_q(\vec{\mathbf{x}}_{j} |\vec{\mathbf{x}}_i, i \in S_f)p(i \in S_f)}{e^{s(\vec{\mathbf{x}}_{j},\vec{\phi}) } + P_q(\vec{\mathbf{x}}_{j}|\vec{\mathbf{x}}_i, i \in S^+_f)}) \Big) \\
  \intertext{Note now that $P_q(\vec{\mathbf{x}}_k|\vec{\mathbf{x}}_i,i\in S^+_f)$ is simply the probability of the record's feature value $\vec{\mathbf{x}}_{k,f}$ under the second step noise distribution $q_2(\mathbf{X_{f}},\alpha_2)$:
   $P_q(\vec{\mathbf{x}}_k|\vec{\mathbf{x}}_i,i\in S^+_f) = P_{q_2}(\vec{\mathbf{x}}_{k,f})$,
   where $k$ refers to record $i$ or a record $j$ negatively sampled from $i$.}\\
=&  \frac{1}{|S^+|}\sum_{f=1}^p \sum_{i \in S^+_f} \Big( \log(\frac{e^{s(\vec{\mathbf{x}}_i,\vec{\phi}) } p(i \in S^+_f)}{e^{s(\vec{\mathbf{x}}_i,\vec{\phi}) } + P_{q_2}(\vec{\mathbf{x}}_{i,f}) } ) + \sum_{\vec{\mathbf{x}}_{j} \sim q(.|\vec{\mathbf{x}}_i,i\in S_f)}^k \log(\frac{  P_{q_2}(\vec{\mathbf{x}}_{j,f})p(i \in S_f)}{e^{s(\vec{\mathbf{x}}_{j},\vec{\phi}) } + P_{q_2}(\vec{\mathbf{x}}_{j,f})) } \Big) \\
   = & \frac{1}{|S^+|}\sum_{f=1}^p \sum_{i \in S^+_f} \Big( \log(\frac{e^{s(\vec{\mathbf{x}}_i,\vec{\phi}) }}{e^{s(\vec{\mathbf{x}}_i,\vec{\phi}) } + P_{q_2}(\vec{\mathbf{x}}_{i,f}) }) + \log(p(i \in S^+_f)^{k+1})\\
  &   +  \sum_{\vec{\mathbf{x}}_{j} \sim q(.|\vec{\mathbf{x}}_i,i\in S^+_f)}^k \log(\frac{ P_{q_2}(\vec{\mathbf{x}}_{j,f})}{e^{s(\vec{\mathbf{x}}_{j},\vec{\phi}) } + P_{q_2}(\vec{\mathbf{x}}_{j,f})})   \Big) \\
\intertext{We now drop the term containing the probability of assignment to a feature type $p(i \in S^+_f)$ since it is outside of the learned model parameters $\vec{\phi}$ and fixed in advance:}\\
    \propto & \frac{1}{|S^+|} \sum_{f=1}^p \sum_{i \in S^+_f} \Big( \log(\frac{e^{s(\vec{\mathbf{x}}_i,\vec{\phi}) }}{e^{s(\vec{\mathbf{x}}_i,\vec{\phi}) } + P_{q_2}(\vec{\mathbf{x}}_{i,f}) }) + \sum_{\vec{\mathbf{x}}_{j} \sim q(.|\vec{\mathbf{x}}_i,i\in S^+_f)}^k \log(\frac{ P_{q_2}(\vec{\mathbf{x}}_{j,f})}{e^{s(\vec{\mathbf{x}}_{j},\vec{\phi}) } + P_{q_2}(\vec{\mathbf{x}}_{j,f})}) \Big) \\
    \xrightarrow[|S^+|\rightarrow \infty]{ }  & \sum_{f=1}^p p(i \in S^+_f) E\Big[ \log(\frac{e^{s(\vec{\mathbf{x}}_i,\vec{\phi}) }}{e^{s(\vec{\mathbf{x}}_i,\vec{\phi}) } + P_{q_2}(\vec{\mathbf{x}}_{i,f}) }) + \sum_{\vec{\mathbf{x}}_{j} \sim q(.|\vec{\mathbf{x}}_i,i\in S^+_f)}^k \log(\frac{ P_{q_2}(\vec{\mathbf{x}}_{j,f})}{e^{s(\vec{\mathbf{x}}_{j},\vec{\phi}) } + P_{q_2}(\vec{\mathbf{x}}_{j,f})}) \Big] \\
   & = \sum_{f=1}^p p(i \in S^+_f) E\Big[ L(\vec{\mathbf{x}} |\vec{\phi}, \text{target = $f$}) \Big]
\end{align*}

Thus, the loss function is just a convex combination  of the loss functions of the targeted classifiers for each of the $p$ features, and by extension so is the gradient since:

\[\frac{\partial }{\partial \phi}\sum_{f=1}^p p(i \in S^+_f) E\Big[ L(\vec{\mathbf{x}} |\vec{\phi}, \text{target = $f$}) \Big] =
 \sum_{f=1}^p p(i \in S^+_f) \frac{\partial }{\partial \phi}E\Big[ L(\vec{\mathbf{x}} |\vec{\phi}, \text{target = $f$}) \Big]\]
 Thus the algorithm will, at each step, learn a convex combination of the gradient for a targeted classifier on feature $f$, with weights proportional to the feature type sampling probabilities in step 1 of the sampling algorithm.
\end{proof}

\subsection{Problems with Flattening Multinomial of $\Word2Vec$}
\label{sec:w2vproblem}
\hl{Write}



\bibliography{joseg}
\bibliographystyle{iclr2018_conference}

\end{document}
