\documentclass{article}

% if you need to pass options to natbib, use, e.g.:
% \PassOptionsToPackage{numbers, compress}{natbib}
% before loading nips_2018

% ready for submission
%\usepackage{aaai}

% to compile a preprint version, e.g., for submission to arXiv, add
% add the [preprint] option:
% \usepackage[preprint]{nips_2018}

% to compile a camera-ready version, add the [final] option, e.g.:
% \usepackage[final]{nips_2018}

% to avoid loading the natbib package, add option nonatbib:
% \usepackage[nonatbib]{nips_2018}

\usepackage[utf8]{inputenc} % allow utf-8 input
\usepackage[T1]{fontenc}    % use 8-bit T1 fonts
\usepackage{hyperref}       % hyperlinks
\usepackage{url}            % simple URL typesetting
\usepackage{booktabs}       % professional-quality tables
\usepackage{amsfonts}       % blackboard math symbols
\usepackage{nicefrac}       % compact symbols for 1/2, etc.
\usepackage{microtype}      % microtypography



%%% custom packages by JPG and LA:
\usepackage[most]{tcolorbox}
\usepackage{amsmath}


\usepackage{upgreek}

\usepackage{soul}
\usepackage{algorithm}
\usepackage{algpseudocode}
\usepackage{booktabs}

\newcommand{\vect}[1]{\vec{#1}}
\DeclareMathOperator{\Feat2Vec}{Feat2Vec}
\DeclareMathOperator{\Word2Vec}{Word2Vec}
\DeclareMathOperator{\q}{{\mathcal{Q}}}
\newcommand{\dotp}{\boldsymbol{\cdot} }
%\renewcommand{\cite}[1]{\citep{#1}}
\usepackage{amsthm}
\usepackage{subcaption}
\usepackage{thmtools,thm-restate}
\declaretheorem{theorem}

\newcommand{\algojose}{Structured Deep-Out Factorization Machine}
\newcommand{\algoralph}{$\Feat2Vec$}
\usepackage{amssymb}
\usepackage[weather]{ifsym}
\usepackage{amsmath}
\usepackage{upgreek}
\usepackage{bm}
\newcommand{\algoralphabbr}{$\Feat2Vec$}




% The \author macro works with any number of authors. There are two
% commands used to separate the names and addresses of multiple
% authors: \And and \AND.
%
% Using \And between authors leaves it to LaTeX to determine where to
% break the lines. Using \AND forces a line break at that point. So,
% if LaTeX puts 3 of 4 authors names on the first line, and the last
% on the second line, try using \AND instead of \And before the third
% author name.
\begin{document}
% \nipsfinalcopy is no longer used
\appendix

\vspace*{-2cm}
\section{Appendix}
Supplemental proof for ``Beyond Word Embeddings: Representations for Multi-modal Data''. by Luis Armona, Jos\'e Gonz\'alez-Brenes, and Ralph Edezath.





\subsection{Proof to Theorem 1}
\begin{theorem}
The gradient for learning embeddings with self-supervised $\Feat2Vec$ is a convex combination of the gradient from $n$ supervised Factorization Machines learned with implicit sampling, one for each feature group in the data.
\label{thm:proof}
\end{theorem}
\begin{proof}
Let $\mathbf{X}$ denote the training data with positive labels, $|\mathbf{X}|$ the number of observations in the data, and $S^+_{\upkappa_i}$ denote the positively labeled records in $\mathbf{X}$ whose corresponding negative samples resample feature group $i$. For convenience, suppress the inclusion of learned parameters $\bm\uptheta$ in the notation in this section while understanding the feature extraction functions $\vec{\phi}$ implicitly include these parameters.
Let $ \tilde y(\vect{x},\vec{\phi}) = \sum_{i =1}^n \sum_{j=i}^n \phi_i( \vect{x}_{\vect{\upkappa}_i}) \dotp \phi_j( \vect{x}_{\vect{\upkappa}_j}) $.
We can express the loss function $L(.)$, the binary cross-entropy of the data given the self-supervised $\Feat2Vec$ model, as follows:

\begin{align*}
 L(\mathbf{X} | \vec{\phi} )  =& -  \frac{1}{|\mathbf{X}|} \sum_{\vect{x}^+ \in \mathbf{X}} \Big(\log(\tilde p(y=1|\vec{\phi},\vect{x}^+  ) ) + \sum_{\vect{x}^{\,-}\sim \q(.|\vect{x}^+)}^k \log(\tilde p(y=0|\vec{\phi},\vect{x}^{\,-})) \Big) \\
  =& - \frac{1}{|\mathbf{X}|} \sum_{\vect{x}^+ \in \mathbf{X}} \Big( \log(\tilde p(y=1|\vec{\phi},\vect{x}^+, \vect{x}^+ \in S^+_{\upkappa_i} )p(\vect{x}^+ \in S^+_{\upkappa_i}) ) \\ &
   + \sum_{\vect{x}^{\,-} \sim \q(.|\vect{x}^+)}^k \log(\tilde p(y=0|\vec{\phi},\vect{x}^{\,-},\vect{x}^+ \in S^+_{\upkappa_i})p(\vect{x}^+ \in S^+_{\upkappa_i})) \Big) \\
  =& - \frac{1}{|\mathbf{X}|}\sum_{i=1}^n \sum_{\vect{x}^+ \in S^+_{\upkappa_i}} \Big( \log(\frac{e^{\tilde y(\vect{x}^+,\vec{\phi}) } p(\vect{x}^+ \in S^+_{\upkappa_i})}{e^{\tilde y(\vect{x}^+,\vec{\phi}) } +k\times P_{\q}(\vect{x}^+ |\vect{x}^+, \vect{x}^+ \in S^+_{\upkappa_i}) }) \\ &
  +  \sum_{\vect{x}^{\,-} \sim \q(.|\vect{x}^+,\vect{x}^+ \in S^+_{\upkappa_i})}^k \log(\frac{k\times P_{\q}(\vect{x}^{\,-} |\vect{x}^+, \vect{x}^+ \in S^+_{\upkappa_i})p(\vect{x}^+ \in S^+_{\upkappa_i})}{e^{\tilde y(\vect{x}^{\,-},\vec{\phi}) } + k\times P_{\q}(\vect{x}^{\,-}|\vect{x}^+, \vect{x}^+ \in S^+_{\upkappa_i})}) \Big) \\
  \intertext{Note now that $P_{\q}(\vect{x}|\vect{x}^+,\vect{x}^+ \in S^+_{\upkappa_i})$ is simply the probability of the record's feature value $\vect{x}_{\upkappa_i}$ under the second step noise distribution $\q_2(\mathrm{X}_{\upkappa_i},\alpha_2)$:
   $P_{\q}(\vect{x}|\vect{x}^+,\vect{x}^+ \in S^+_{\upkappa_i}) = P_{\q_2}(\vect{x}_{\upkappa_i})$}\\
=& - \frac{1}{|\mathbf{X}|}\sum_{i=1}^n \sum_{\vect{x}^+ \in S^+_{\upkappa_i}} \Big( \log(\frac{e^{\tilde y(\vect{x}^+,\vec{\phi}) } p(\vect{x}^+ \in S^+_{\upkappa_i})}{e^{\tilde y(\vect{x}^+,\vec{\phi}) } + k\times P_{\q_2}(\vect{x}^+_{\upkappa_i}) } ) + \sum_{\vect{x}^{\,-} \sim \q(.|\vect{x}^+,i\in S^+_{\upkappa_i})}^k \log(\frac{ k\times P_{\q_2}(\vect{x}^{\,-}_{\upkappa_i})p(\vect{x}^+ \in S^+_{\upkappa_i})}{e^{\tilde y(\vect{x}^{\,-},\vec{\phi}) } + k\times P_{\q_2}(\vect{x}^{\,-}_{\upkappa_i})) } \Big) \\
   = & - \frac{1}{|\mathbf{X}|}\sum_{i=1}^n \sum_{\vect{x}^+ \in S^+_{\upkappa_i}} \Big( \log(\frac{e^{\tilde y(\vect{x}^+,\vec{\phi}) }}{e^{\tilde y(\vect{x}^+,\vec{\phi}) } + k\times P_{\q_2}(\vect{x}^+_{\upkappa_i}) }) + \log(p(\vect{x}^+ \in S^+_{\upkappa_i})^{k+1})\\
  &   +  \sum_{\vect{x}^{\,-} \sim \q(.|\vect{x}^+,\vect{x}^+ \in S^+_{\upkappa_i})}^k \log(\frac{ k\times P_{\q_2}(\vect{x}^{\,-}_{\upkappa_i})}{e^{\tilde y(\vect{x}^{\,-},\vec{\phi}) } +k\times P_{\q_2}(\vect{x}^{\,-}_{\upkappa_i})})   \Big) \\
\intertext{We now drop the term containing the probability of assignment to a feature group $p(\vect{x}^+ \in S^+_{\upkappa_i})$ since it is outside of the learned model parameters $\vec{\phi}$ and fixed in advance:}\\
    \propto & -\frac{1}{|\mathbf{X}|} \sum_{i=1}^n \sum_{\vect{x}^+ \in S^+_{\upkappa_i}} \Big( \log(\frac{e^{\tilde y(\vect{x}^+,\vec{\phi}) }}{e^{\tilde y(\vect{x}^+,\vec{\phi}) } +k\times P_{\q_2}(\vect{x}^+_{\upkappa_i}) }) + \sum_{\vect{x}^{\,-} \sim \q(.|\vect{x}^+,\vect{x}^+ \in S^+_{\upkappa_i})}^k \log(\frac{k\times  P_{\q_2}(\vect{x}^{\,-}_{\upkappa_i})}{e^{\tilde y(\vect{x}^{\,-},\vec{\phi}) } +k\times P_{\q_2}(\vect{x}^{\,-}_{\upkappa_i})}) \Big) \\
    \xrightarrow[|\mathbf{X}|\rightarrow \infty]{ }  & -\sum_{i=1}^n p(\vect{x}^+ \in S^+_{\upkappa_i}) E\Big[ \log(\frac{e^{\tilde y(\vect{x}^+,\vec{\phi}) }}{e^{\tilde y(\vect{x}^+,\vec{\phi}) } +k\times P_{\q_2}(\vect{x}^+_{\upkappa_i}) }) + \sum_{\vect{x}^{\,-} \sim \q(.|\vect{x}^+,\vect{x}^+ \in S^+_{\upkappa_i})}^k \log(\frac{ k\times P_{\q_2}(\vect{x}^{\,-}_{\upkappa_i})}{e^{\tilde y(\vect{x}^{\,-},\vec{\phi}) } +k\times P_{\q_2}(\vect{x}^{\,-}_{\upkappa_i})}) \Big] \\
   & = \sum_{i=1}^n p(\vect{x}^+ \in S^+_{\upkappa_i}) E\Big[ L(\vec{x} |\vec{\phi}, \text{target = $i$}) \Big]
\end{align*}

Thus, the loss function is just a convex combination  of the loss functions of the targeted classifiers for each of the $p$ features, and by extension so is the gradient since:

\[\frac{\partial }{\partial \phi}\sum_{i=1}^n p(\vect{x}^+ \in S^+_{\upkappa_i}) E\Big[ L(\vec{x} |\vec{\phi}, \text{target = $i$}) \Big] =
 \sum_{i=1}^n p(\vect{x}^+ \in S^+_{\upkappa_i}) \frac{\partial }{\partial \phi}E\Big[ L(\vec{x} |\vec{\phi}, \text{target = $i$}) \Big]\]
 Thus the algorithm will, at each step, learn a convex combination of the gradient for a targeted classifier on feature group $i$, with weights proportional to the feature group sampling probabilities in step 1 of the sampling algorithm $ p(\vect{x}^+ \in S^+_{\upkappa_i})$. 
\end{proof}







\end{document}
