\def\year{2019}\relax
\documentclass[letterpaper]{article}
\usepackage{aaai19}
\usepackage{times}
\usepackage{helvet}
\usepackage{courier}
\usepackage{graphicx}
\frenchspacing
\setlength{\pdfpagewidth}{8.5in}
\setlength{\pdfpageheight}{11in}
\setcounter{secnumdepth}{2}

\usepackage[utf8]{inputenc} % allow utf-8 input
\usepackage[T1]{fontenc}    % use 8-bit T1 fonts
\usepackage{url}            % simple URL typesetting
\usepackage{booktabs}       % professional-quality tables
\usepackage{amsfonts}       % blackboard math symbols
\newcommand{\citet}[1]{\citeauthor{#1} \shortcite{#1}}
\newcommand{\citep}{\cite}
\newcommand{\citealp}[1]{\citeauthor{#1} �\citeyear{#1}}

\usepackage{upgreek}
\usepackage{amsmath}
\usepackage{soul}
\usepackage{algorithm}
\usepackage{algpseudocode}
\usepackage{booktabs}

\newcommand{\vect}[1]{\vec{#1}}
\DeclareMathOperator{\Feat2Vec}{Feat2Vec}
\DeclareMathOperator{\Word2Vec}{Word2Vec}
\DeclareMathOperator{\Doc2Vec}{Doc2Vec}
\DeclareMathOperator{\q}{{\mathcal{Q}}}
\newcommand{\dotp}{\boldsymbol{\cdot} }
\usepackage{subcaption}

\newcommand{\algojose}{Structured Deep-Out Factorization Machine}
\newcommand{\algoralph}{$\Feat2Vec$}
\usepackage{amssymb}
\usepackage[weather]{ifsym}
\usepackage{amsmath}
\usepackage{upgreek}
\usepackage{bm}
\usepackage{mathtools}
\newcommand{\algoralphabbr}{$\Feat2Vec$}
%
%
\pdfinfo{
/Title (Beyond Word Embeddings: Dense Representations for Multi-modal Data)
/Author (Luis Armona, Ralph Edezath, Jos\'e P. Gonz\'alez-Brenes)
}

\title{Beyond Word Embeddings: Dense Representations for Multi-modal Data}

\author{
  Luis Armona\textsuperscript{\rm1,\rm 2},  Jos\'e P. Gonz\'alez-Brenes,\textsuperscript{\rm 2}\thanks{Equal contribution} Ralph Edezhath\textsuperscript{\rm 2}\footnotemark[1] \\
  \textsuperscript{\rm 1}Department of Economics, Stanford University, \\
  \textsuperscript{\rm 2}Chegg, Inc.\\
  larmona@stanford.edu, \{redezhath, jgonzalez\}@chegg.com
}

\begin{document}
\maketitle
\begin{abstract}
Methods that calculate dense vector representations for text have proven to be very successful for knowledge representation.
We study how to estimate dense representations for multi-modal data (e.g., text, continuous, categorical).
We propose Feat2Vec as a novel model that supports supervised learning when explicit labels are available, and self-supervised learning when there are no labels.
Feat2Vec calculates embeddings for data with multiple feature types, enforcing that all embeddings exist in a common space.
We believe that we are the first to propose a method for learning self-supervised embeddings that leverage the structure of multiple feature types.
Our experiments suggest that Feat2Vec outperforms previously published methods, and that it may be useful for avoiding the cold-start problem.
\end{abstract}



\section{Introduction}

Informally, in machine learning a \textit{dense representation}, or \textit{embedding}  of a vector   $\vect{x} \in \mathbb{R}^n$  is another vector $\vect{\beta} \in \mathbb{R}^r$ that has much lower dimensionality ($r \ll n$) than the original representation.
In general, we consider two kind of models that produce embeddings:
\textbf{(i)~supervised methods}, like matrix factorization, calculate embeddings that are highly tuned to a prediction task.
For example, in the Nextflix challenge, movie identifiers  are embedded to predict user ratings.
On the other hand, \textbf{(ii)~self-supervised methods} (sometimes referred to as unsupervised methods) are not tuned for the prediction task they are ultimately used for.
For example, word embedding algorithms such as $\Word2Vec$~\cite{mikolov2013distributed} are self-supervised. These algorithms are typically evaluated by analogy solving, or sentiment analysis~\cite{le2014distributed}, even though their loss functions are not tuned for either of these tasks.

Throughout this paper, we refer to any data that has different types of information as multi-modal.
We propose $\Feat2Vec$ as a novel method that  embeds multi-modal data, such as text, images, numerical or categorical data, in a common vector space---for both supervised and self-supervised scenarios.
We believe that this is the first self-supervised algorithm that is able to calculate embeddings from data with multiple feature types.
Consider the non-trivial work that was required to extend a model like Word2Vec to support  additional features.
The authors of the seminal Doc2Vec~\cite{le2014distributed} paper needed to design both a new neural network and a new sampling strategy  to add a single feature (document ID).
$\Feat2Vec$ is a general method that allows calculating embeddings for any number of features.

\section{Feat2Vec}
\label{sec:methods}
In this section we describe how $\Feat2Vec$  learns embeddings of feature groups.

\subsection{Model}
\label{sec:algoralph}
$\Feat2Vec$ predicts  a target output  $\tilde y$ from each observation ${\vect{\mathbf x}}$, which is multi-modal data that can be interpreted as a list of feature groups $\vect{g_i}$:
\begin{align}
%  {\vect{\mathbf x}} =   \langle \vect{x}_{\vect \upkappa_1}, \vect{x}_{\vect \upkappa_2}, \dots, \vect{x}_{\vect \upkappa_n} \rangle  = \langle \vect{g}_1, \vect{g}_2, \dots, \vect{g}_n \rangle
  {\vect{\mathbf x}} =   \langle \vect{g}_1, \vect{g}_2, \dots, \vect{g}_n \rangle  = \langle \vect{x}_{\vect \upkappa_1}, \vect{x}_{\vect \upkappa_2}, \dots, \vect{x}_{\vect \upkappa_n} \rangle
 \end{align}
For notational convenience, we also refer to the $i$-th feature group as  $\vect{x}_{\vect \upkappa_i}$.
In other words, each observation is constructed from a partition of raw features specified by $\vect\upkappa$.
Examples of  how to group features include ``text'' from individual words,  or an ``image'' from individual pixels.
We define an entity to be a particular value or realization of a feature group.

We define our model for a target $y$ as follows:
\begin{align}
\tilde y(\vect{{x}\vphantom{phi}},\vec{\pmb{\upphi}}, \vect{\pmb{\uptheta}\vphantom{phi}} ) = \omega \biggl(
                       \sum_{i =1}^n \sum_{j=i}^n  \overbracket {\phi_i( \vect{x \vphantom{\uptheta} }_{\vect{\upkappa}_i};  \vect{{\uptheta}}_i) }^{ \mathclap  { \text{group $i$}\atop \text{embedding}}} \dotp \overbracket {\phi_j( \vect{x \vphantom{\uptheta} }_{\vect{\upkappa}_j}; \vect{{\uptheta}_j  }) }^{ \mathclap  { \text{group $j$}\atop \text{embedding}}}
  \biggr)
 \label{eq:deepfm}
\end{align}

The number of dimensions $d_i$ of each feature group may vary, but all of them are embedded to the same space via their \textit{feature extraction function} $\phi_i : \mathbb{R}^{d_i} \rightarrow \mathbb{R}^{r}$.
These functions learn how to transform each input (feature group) into an $r$-dimensional embedding of length $r$. %in $\mathbb{R}^r$.
We only allow a feature extraction function $\phi_i$ to act on the features of $g_i$; and feature groups interact with each other only via the output of $\phi_i$.
We denote the parameters of the extraction function $\phi_i$ as $\vect{{\uptheta}}_i$.
$\omega$ is an activation (link) function.
Intuitively, the dot product ($\dotp$) returns a scalar that measures the (dis)similarity between the embeddings of the feature groups.

A simple  implementation of $\phi_i$ is a linear fully-connected layer, where the output of the $r$-th entry is:
\begin{equation}
\label{eq:fully_connected}
\phi_{i}\bigl( \vect{x\vphantom{\theta}}_i; \vect{\uptheta}_{i}\bigr )_r= \sum_{a=1}^{d_i}  \theta_{{i_r}_a} x_{i_a}
\end{equation}
where $ \vect{\uptheta}_{i,r} $ are learned weights. Other functional forms of $\phi_i$ we have explored in our experiments include:
\begin{itemize}
\item A convolutional neural network that transforms a text sequence into an $r$-dimensional embedding. This function is able to preserve information in the order of a text sequence, unlike other methods for transforming text to numerical values such as bag-of-words.
\item A deep, fully connected neural network that projects a scalar, such as average rating, into an $r$-dimensional embedding space. This extraction function, by treating the input as numerical instead of categorical, requires inputs close in value (e.g. 4.01 and 4.02 star rating) to have similar embeddings.
\end{itemize}
Any neural function that outputs an $r$-dimensional vector could be plugged into our framework as an extraction function $\phi_i$.
Although in this work we have only used datasets with natural language, categorical and numerical features, it would be straightforward to incorporate other types of data--e.g. images or audio--with the appropriate feature extraction functions, enabling the embedding of multi-modal features in the same space.

Figure~\ref{fig:architecture} compares existing factorization methods  with our novel model.
Figure~\ref{fig:factorization_machine} shows a Factorization Machine~\cite{rendle2010factorization} that embeds each feature.
Figure~\ref{fig:feat2vec}  shows $\Feat2Vec$ with two feature groups:
the first group only has a single feature which is projected to an embedding (just like matrix factorization);
but the second group has multiple features, which are together projected to a single embedding.
Figure~\ref{fig:neural} shows an approach of using neural networks within factorization machines  that has been proposed multiple times~\cite{DBLP:journals/corr/DziugaiteR15,guo2017deepfm}.
It replaces the dot product of factors with a learned neural function, which has been shown to improve predictive accuracy.
The caveat of this architecture is that is no longer  possible to interpret the embeddings as latent factors related to the target task.

\begin{figure*}[th]
    \centering
    \begin{subfigure}[t]{0.26\textwidth}
        \centering
        \includegraphics[width=.55\textwidth]{img/diagram_mf.pdf}
        %\includegraphics[height=1in]{img/diagram_mf.pdf}
        \caption{\textbf{Factorization Machine}\label{fig:factorization_machine}}
    \end{subfigure}%
    ~
    \begin{subfigure}[t]{0.26\textwidth}
        \centering
        \includegraphics[width=.5\textwidth]{img/diagram_in.pdf}
        \caption{\textbf{Feat2Vec} \label{fig:feat2vec}}
    \end{subfigure}
    ~
    \begin{subfigure}[t]{0.26\textwidth}
        \centering
        \includegraphics[width=.6\textwidth]{img/diagram_nmf.pdf}
        \caption{\textbf{``Neural"} \label{fig:neural}}
    \end{subfigure}
    \caption{\textbf{Network architectures for factorization models}.  The  white clouds (\Cloud) represent deep layers, for example a convolutional network for text features, while the dots ($\cdot$) denote dot products.
    \label{fig:architecture}
    }
\end{figure*}


Our work in $\Feat2Vec$  significantly expands previously published models.
For example, Principal Component Analysis (PCA) is a common method to learn embeddings of individual dimensions of  data.
Although structured regularization techniques exist~\cite{jenatton2010structured}, it is not obvious how to combine different dimensions in PCA when we are interested in treating a subset of dimensions as a single group.
In traditional factorization methods, item identifiers are embedded and thus need to be observed during training (i.e., cold-start problem).
$\Feat2Vec$ can learn an embedding from an alternative characterization of an item, such as a textual description.
$\Feat2Vec$ extends Factorization Machine~\cite{rendle2010factorization}, in that we allow calculating embeddings for feature groups and we use feature extraction functions.
StarSpace~\cite{wu2017starspace}  introduces feature groups (called entities in their paper), but constrains  them to be ``bags of features,'' is supervised, and limited to two feature types.
In contrast, $\Feat2Vec$ allows continuous features, and does not require the user to assign one feature as a specific target. Additionally, we propose a self-supervised learning algorithm.


\subsection{Supervised Learning from Data}
\label{sec:learning_supervised}

We can learn the the parameters of a Feat2Vec model $  \vect{\bm\uptheta}$  by minimizing a supervised loss function $\mathcal L_\text{sup}$:
\label{sec:learning}
\begin{equation}
\arg \min_{ \vect{\bm\uptheta}} \sum_{\vect{x} \in \mathbf{X}}  \mathcal{L}_\text{sup}\bigl( \, y(\vect{x}), \tilde y(\vect{x};{\bm \upphi,  \vect{\bm\uptheta}}) \, \bigr)
%+ \gamma||  \bm{\uptheta}||^{w}
\label{eq:learning}
\end{equation}
Here, $\mathbf X$ is the training dataset, $y(\vect{x})$  is the true target value observed in the data for an observation $\vect{x} \in \mathbf{X}$, and $\tilde {y}(\vect{x})$ is its estimated value (using Equation \ref{eq:deepfm});
$ \vect{\bm\uptheta}$ represents the parameters learned  during training (i.e. the parameters associated with the extraction functions $\phi_i$).
%The hyperparameter $\gamma$ controls the amount of regularization.
For labeling and classification tasks, we optimize the logistic loss.

We can optimize Equation~\ref{eq:learning} directly using gradient descent algorithms.
Multi-class classification models can be computationally costly to learn when the number of classes is very high.
In these cases, a common work-around is to  reformulate the learning problem into a binary classifier with implicit sampling~\cite{samplingnotes}:
simply redefine the observation by appending the target label  as one of the feature groups of the observation (e.g., rating), while keeping the other feature group(s) the same (e.g. review text).
The new implicit target label is a binary value that indicates whether the observation exists in the training dataset.

In implicit sampling, instead of using all of the possible negative labels, one samples a fixed number ($k$) from the set of possible negative labels for each positively labelled record.
Implicit sampling is also useful for multi-label classification with a large number of labels, or when negative examples are not easily available.
We leverage on implicit sampling for our self-supervised learning algorithm.

\subsection{Self-Supervised Learning From Data}
\label{sec:sampling}

We now discuss how $\Feat2Vec$ can be used to learn embeddings in an self-supervised setting with no explicit target for prediction.


The training dataset for a $\Feat2Vec$ model consists of the observed data.
In natural language, these would be documents written by humans, along with document metadata.
Since self-supervised Feat2Vec will require positive and negative examples during training, we supply unobserved data as negative examples.
This sampling procedure is analogous to the $\Word2Vec$ algorithm, which samples a negative observation from a noise distribution $\q_{w2v}$ that is proportional to the empirical frequency of a word in the training data.
Unlike $\Word2Vec$, we do not constrain features types to be words.
Instead, by leveraging the feature groups of the data defined by $\vect{\pmb{\upkappa}}$ (Equation \ref{eq:deepfm}), the model can reason on more abstract entities in the data.
For example, in our experiments on a movie dataset, we define a ``genre'' feature group, where we  group non-mutually exclusive indicators for movie genres, including comedy, action, and drama films.

We start with a training dataset $\mathbf X $ of records with $n$ feature groups. We assign a positive label to each observation in $\mathbf X$.
For each observed (positive) datapoint $\vect{x}^+$, we generate $k$ observations with negative labels using the 2-step algorithm documented in Algorithm~\ref{algo:samp}.
We illustrate the intuition of this algorithm with an example.
Consider a dataset with three feature groups---entities for (i) a passage of text (bag of words), (ii) a restaurant identifier (categorical), and a sentiment (a continuous value).
The positive examples are observations seen in the training set.
Negative examples are generated by modifying positive examples, specifically by substituting one of the entities randomly from another observation.
For example, by substituting the passage of text, a negative example for the observation $\langle$\textit{Food was great.}, \textsl{Le Fancy Restaurant}, $100\rangle$ may result as
$\langle$\textit{Ewww, gross!!!}, \textsl{Le Fancy Restaurant}, $100\rangle$.


\begin{algorithm}[htb]
 \caption{Implicit sampling algorithm for self-supervised $\Feat2Vec$ \label{algo:sampling}}
 \label{algo:samp}
\begin{algorithmic}[1]
\footnotesize
\Function{$\q$}{$\vect{x}^{\,+}; k, \alpha_1, \alpha_2$}
\State $ \mathbf{X}^-  \leftarrow \emptyset $
%\For {$\vect{x}^{\,+} \in X^+ $}
	\For {$j \in \{1,\ldots,k\}$}
			\State Draw  which feature group $i \sim \q_1(\bm{\upphi},\alpha_1) $ to sample
			\State  Draw a random entity $\vect{g} \sim \q_2(\mathrm{X_{\upkappa_i}}, \alpha_2)$
			\State $\vect{x}^{\,-}  \leftarrow \vect{x}^{\,+} $ \Comment  Set identical to the positive sample
			\State $\vect{x}^{-}_{\upkappa_i} \leftarrow \vect{g}$ \Comment{substitute the $i$-th feature group with $\vect{g}$}
			\State  $\mathbf{X}^-  \leftarrow \mathbf{X}^-  + \{  \vect{x}^{\,-} \}$ \Comment{append}
	\EndFor
%\EndFor
\State \Return   $\mathbf{X}^-$ \Comment{Sampled negative observations from $\vect{x}^{\,+}$}
\EndFunction
\end{algorithmic}
\end{algorithm}


More formally, our negative sampling algorithm generates an observation with a negative label from an observation in the training dataset as follows.
First, it randomly selects the $i$-th feature group from a noise distribution $\q_1(\cdot)$.
It then creates a negative observation that is identical to $\vect{x}^{\,+}$, except that its $i$-th feature group value is replaced by a value sampled from a noise distribution $\q_2(\cdot)$.
In our application, we use the same class of noise distributions (flattened multinomial) for both levels of sampling, but this need not be the case.
We now describe the noise distributions that we use.

\textbf{Sampling Feature Groups.} The function $\operatorname{params}$ calculates the complexity of a  feature extraction function $\phi_i$.
To sample a feature group, we choose a feature group $i$ from a multinomial distribution with probabilities proportional to a feature group's complexity.
By complexity, we mean the number of parameters we learn that are associated with a feature group's extraction function $\phi_i$.
This sampling procedure places more weight on features that have more parameters and thus are going to require more training iterations to properly learn. The sampling probabilities of each feature group are:

\begin{equation}
P_{\q_1}(i|\bm{\upphi},\alpha_1 )  \
= \frac{\operatorname{params}(\phi_i)^{\alpha_1}}{\sum_{j=1}^{|\vect{\upkappa}|} \operatorname{params}(\phi_j)^{\alpha_1}}
\end{equation}
For categorical variables using a linear fully-connected layer, the complexity is simply proportional to the number of categories in the feature group.
However, if we have multiple intermediate layers for some feature extraction functions (e.g., convolutional layers), these parameters should also be counted towards a feature group's complexity.
The hyper-parameter $\alpha_1 \in [0,1]$ flattens the distribution.
 When  $\alpha_1=0$, the feature groups are sampled uniformly,
 and when $\alpha_1=1$, they are sampled exactly proportional to their complexity.


\textbf{Sampling Feature Group Values.}
To sample an entity within a feature group $i$, we use a similar strategy to $\Word2Vec$ and use the empirical distribution of values:
%For  $\q_2(\mathrm{X}_{\upkappa_i}, \alpha_2)$, is the ``flattened'' empirical multionomial distribution over the frequency of the feature values of feature $\upkappa_i$, as in the negative sampling procedure used in $\Word2Vec$ :
\begin{equation}
P_{\q_2}(\vect{g} |\mathrm{X}_{\upkappa_i},\alpha_2)  \
= \frac{ \operatorname{count}(\vect{g} )^{\alpha_2}}{\sum\limits_{\vect{j} \in\mathrm{X}_{\upkappa_i} } \operatorname{count}(\vect{j})^{\alpha_2}}, \quad \alpha_2 \in [0,1]
\end{equation}
Here, $\operatorname{count}(\vect{g})$ is the number of times an entity $\vect{g}$ of feature group $i$ appeared in the training dataset $\mathbf{X}$ .
Thus, an entity is sampled according to its empirical distribution---whether it is a single word, a continuous number, or a passage of text.
Again, $\alpha_2$ is simply a flattening hyperparameter.


This method will sometimes by chance generate negatively labeled samples that \textit{do} exist in our sample of observed records.
The literature offers two solutions:
in the Negative Sampling of $\Word2Vec$, duplicate negative samples are ignored~\cite{samplingnotes}.
Instead, we account for the probability of random negative labels being identical to positively labeled data using \textit{Noise Contrastive Estimation} (NCE)~\cite{nce}.

%%%%%%%%%%%%%%%%
\subsubsection{The Loss Function for Self-Supervised Learning}
\label{sec:learning}
For our self-supervised learning of embeddings, we optimize a NCE loss function.
We need to make sure that  our loss function is well-behaved distribution that integrates to 1.
This often requires a partition function $Z_{\vec{x}}$ for each unique record type $\vec{x}$,
which can be computationally costly and greatly increase the complexity of our model.
Instead, we appeal to the work of \citet{fastnnlang},
who show that in models with many parameters setting $Z_{\vec{x}}=1$ in advance does not change the performance of the model.
The intuition is that if the underlying model has enough free parameters, it will effectively learn the probabilities itself, since systemic under/over prediction of probabilities will result in penalties on the loss function.

For self-supervised learning, we adjust the structural statistical model of Equation~\ref{eq:deepfm} to account for the probability of negative examples, as in \cite{samplingnotes}.
Written explicitly, this is:
\begin{align}
\tilde p(y=1|\vec{x\vphantom{phi}},\vect{\bm\upphi},\vect{\bm\uptheta}) &=
 \frac{\exp\bigl(\tilde y (\vec{x\vphantom{phi}}|\vect{\bm\upphi},\vect{\bm\uptheta}) \bigr)}{\exp( \tilde y \bigl(\vec{x\vphantom{phi}}, \vect{\bm\upphi},\vect{\bm\uptheta}) \bigr) + k\times P_{\q}(\vec{\vphantom{phi}x}| \alpha_1,\alpha_2)} \nonumber \\
 & \triangleq \hat s(\vect{x\vphantom{\uptheta}};\vect{\bm\uptheta})
\end{align}
For notational convenience, we refer to this structural probability as $\hat s(\vect{x\vphantom{\uptheta}}; \vect{\bm\uptheta})$.
Here, $\tilde y (.)$ denotes the model in Equation $\ref{eq:deepfm}$ with $\omega$ set to the identity function. This can be interpeted as the ``score'' of an observation with respect to $\Feat2Vec$.
 $P_{\q}(x)$ denotes the probability of $x$ under noise distribution $\q$;
 it accounts for the possibility of sampling an observation with a negative label from Algorithm \ref{algo:samp} that appears identical to one in the training data (with a positive label). The probability of a negatively labeled record $\vec{x}$, conditional on a positive datapoint $\vec{x}^+$, is simply given by:
\begin{align}
P_{\q}(\vec{x}|\alpha_1,\alpha_2,\mathbf{X},\vect{x}^+) = P_{\q_2}(\vec{x}_{\upkappa_i}|\mathrm{X}_{\upkappa_i},\alpha_2)\times \nonumber \\
                                                                                            P_{\q_1}(i|\operatorname{params}(\phi_i)\}_{i=1}^n,\alpha_1)
\end{align}
%The details of the sampling of observations with negative labels is explained in \S~\ref{sec:sampling}.

We seek to optimize the parameters $ \vect{\bm\uptheta}$ of the feature extraction functions $\vec{\pmb{\upphi}}$:
\begin{equation}
\arg\min_{ \vect{\bm\uptheta}} \sum_{\vect{x}^{\,+} \in \mathbf{X}}  \mathcal{L}_\text{self} \bigl( {\vect{x\vphantom{uptheta}}^{\,+}};  \vect{\bm\uptheta}  \bigr)
\end{equation}
For this, we define our self-supervised loss $\mathcal{L}_\text{self}$ as:
\begin{align}
-\sum_{\vect{x}^{\,+} \in \mathbf{X}} \biggl( \log\bigl(\hat s(\vect{x\vphantom{uptheta}}^{\,+}; \vect{\bm\uptheta} )   \bigr) +  \textstyle\sum\limits_{\vect{x}^{\,-}  \sim \q(\cdot|\vect{x}^+)}^k \log\bigl(1-\hat s(\vect{x\vphantom{uptheta}}^{\,-}; \vect{\bm\uptheta}) \bigr) \biggr)
\end{align}

Note that the observations with positive labels come directly from the training data, and the observations with negative labels are sampled using Algorithm~\ref{algo:sampling}.
With $n$ feature groups, the loss function of self-supervised $\Feat2Vec$ can be shown to be a weighted average of the losses from $n$ distinct supervised $\Feat2Vec$ models,\footnote{The proof can be found in supplementary material at \url{http://web.stanford.edu/~larmona/feat2vec/supplemental_theorem.pdf}} one for each feature group.
In other words, it optimizes $n$ multi-label classifiers, where each classifier is optimized for a different feature group. The intuition is as follows. In Algorithm \ref{algo:samp}, we first sample a target feature group $i$ according to $P_{\q_1}$, and then add the loss from a supervised $\Feat2Vec$ model for $i$ to the total self-supervised loss. Thus, $P_{\q_1}$ determines the weights the self-supervised loss function assigns to each feature group.


\section{Empirical Results}
\label{sec:evaluation}
For all our experiments, we define a development set and a single test set which is 10\% of the dataset, and a part of the development set is used for early stopping or validating hyper-parameters.
% In preliminary experiments we noticed that regularization slows down convergence with no gains in prediction accuracy, so we avoid overfitting only by using early stopping.

\subsection{Supervised Embeddings}
\label{sec:targeted embeddings}

To evaluate our supervised learning model, we compare against two baselines:
(i) a method called DeepCoNN~\cite{deepconn},which is  a deep network specifically designed for  incorporating text into recommendation problems---reportedly,  it is the state of the art for predicting customer ratings when textual reviews are available;
(ii) and with Matrix Factorization \cite{koren_matrixfactorization}, a commonly used baseline.

We use our own implementation of Matrix Factorization,
but we rely on DeepCoNN's published results  (at time of writing, their  implementation was not available).
Fortunately, DeepCoNN also compares on Matrix Factorization, which makes interpretation of our results easier.
To make results comparable, the $\Feat2Vec$ experiments use the same feature extraction reported  by DeepCoNN.
Thus, we can rule out that differences of performance are due to choices in feature extraction functions.
Instead of tuning the hyper-parameters, we follow previously published  guidelines~\cite{zhang2015sensitivity}.

We evaluate  on the Yelp dataset\footnote{\url{https://www.yelp.com/dataset/challenge}}, which consists of 4.7 million reviews of restaurants.
For each user-item pair, DeepCoNN concatenates the text from all reviews for that item and all reviews by that user.
The concatenated text is fed into a feature extraction function followed by a factorization machine.
For $\Feat2Vec$, we build 3 feature groups: item (restaurant) identifiers, user identifiers, and review text.

Table \ref{ratingpred} summarizes our results.
We use mean squared error (MSE) as the evaluation metric.
\algoralph{} provides a large performance increase.
Additionally, we claim that 
our approach is more general, because it can trivially be extended to incorporate more features.
It is also  more efficient:
DeepCoNN relies on concatenating text, and when the average reviews per user is $\bar{n_u}$  and the average reviews per item is $\bar{n_i}$, this results in text duplicated on average $\bar{n_i}\times \bar{n_u}$ times per training epoch.
In contrast, for \algoralphabbr{} each review is seen only once per epoch.
Thus \algoralphabbr{}, can be 1-2 orders of magnitude more efficient for datasets where $\bar{n_i}\times \bar{n_u}$ is large.


\begin{table}[ht]
\centering
\caption{Supervised Yelp rating prediction}
\label{ratingpred}
\begin{tabular}{@{}lcc@{}}
\toprule
                     & MSE   & Improvement over MF \\ \midrule
Feat2Vec             & \textbf{0.480} &  \textbf{69.2} \%                                 \\
DeepCoNN             & 1.441 & 19.6 \%                                \\
MF (Matrix   & 1.561 & -                                      \\
Factorization) \\
\bottomrule
\end{tabular}
\end{table}



\subsection{Self-Supervised Embeddings}
\label{sec:unsupexp}


The prediction of similar or related words is a commonly used method for evaluating self-supervised word embeddings~\cite{bojanowski2016enriching,mikolov2013efficient}.
We generalize this task for feature types beyond text as follows.
Given a dataset with $n$ feature groups  we have trained a self-supervised $\Feat2Vec$ model on, we choose a single group $i$ as the ``target'' and a group $j$ as the predictor.
Given a held-out instance $\vect{x}$, we rank all possible values of the target group in $\mathrm{X}_{\upkappa_i}$  by cosine similarity to the source group's embedding $\phi_j( \vect{x}_{\vect{\upkappa}_j})$ for that instance. We then retrieve the ranking of the actual target entity in the test dataset relative to cosine similarity scores for all other possible entities in $\vect{\upkappa}_i$.
Rankings are evaluated according to  their  mean percentile rank (MPR):
$MPR = \frac{1}{N}\sum_{i=1}^N R_i/(\max_i R_i)$, where $R_i$ is the rank of the entity for observation $i$ under our evaluation procedure, and $\max_i R_i$ is the worst ranking that can be received.
A score of 0 would indicate perfect performance (i.e. top rank every test sample given), so lower is better under this metric.

Under this evaluation, we will compare the performance of $\Feat2Vec$ algorithm to $\Word2Vec$'s CBOW algorithm for learning embeddings. During training, CBOW uses a ``context window'' of nearby words to predict a target word, so there is a natural way to apply embeddings learned from CBOW to our ranking task (aggregate context embeddings in the left-out instance's source group to predict the target group).
To create a corpus for training $\Word2Vec$,  each record in the data is treated as a separate document.
Specifically, we tokenize all feature values in a record and treat them as words (e.g. a movie record flagged as belonging to the horror genre has the token  \texttt{genre\_horror} in its document).
We set the $\Word2Vec$ context window wide enough so that training is invariant to the token order.
We use hyperparameters $\alpha_2=.75$ and $k=5$ from \cite{mikolov2013distributed} and set the embeddings dimension to $r=50$. These hyperparameters are used in each of our embedding models during evaluation.
The novel feature group weighting parameter in $\Feat2Vec$ was set to $\alpha_1=0.25$ in order to highlight the differences with $\Word2Vec$, since $\alpha_1=1$ corresponds to all feature groups being sampled equally.
This parameter $\alpha_1$ was not tuned further.

We evaluate self-supervised $\Feat2Vec$  on 3 datasets:


\begin{itemize}
\item\textbf{Movies}
The Internet Movie Database (IMDB) is a publicly available dataset\footnote{\url{http://www.imdb.com/interfaces/}} of information related to films, television programs and video games.
We limit our experiments to data on its 465,136 movies.
Table \ref{tab:features} summarizes our feature groups. For groups with multiple categories, we use a ``Bag of categories'' approach. For example, we sum embeddings of individual actors such as Tom Cruise to produce a single ``actor'' feature group embedding associated with a movie.


\item\textbf{Education}
We use a dataset from a leading technology company that provides educational services.
In this proprietary dataset, we have 57  million observations and 9 categorical feature types which include textbook identifier, user identifier, school identifier,
along with other proprietary features.
Each observation is an interaction a user had with a textbook.

\item\textbf{Yelp}
We use the Yelp dataset from our supervised experiments to evaluate the efficacy of self-supervised embeddings in ratings prediction.
We train embeddings for the following feature groups: user ID, restaurant ID, review text, number of times a review is flagged as funny, and 0-5 star rating of the review.
\end{itemize}

\begin{table}
\centering
\footnotesize
\caption{IMDB Feature Groups}
\label{tab:features}
\begin{tabular}{@{}lllp{6cm}@{}}
\toprule
Feature Type Name      & Type              & \# of features  \\ \midrule
Runtime (minutes)       & Real-valued       & 1          \\
IMDB rating (0-10)      & Real-valued       & 1          \\
\# of rating votes & Real-valued       & 1           \\
Is adult film?          & Boolean           & 2         \\
Movie release year      & Categorical          & 271      \\
Movie title             & Text      & 165,471       \\
Directors               & Bag of categories & 174,382                          \\
Genres                  & Bag of categories & 28                  \\
Writers                 & Bag of categories & 244,241             \\
Principal actors & Bag of categories & 1,104,280      \\\bottomrule
\end{tabular}
\end{table}



\subsubsection{Results}

After training IMDB embeddings, we use cast members in a held out set of movies to  predict the actual director of the film.
We rank directors by cosine similarity of their embedding to the title's cast member feature group embedding.
For the educational dataset, we directly  retrieve the most similar textbooks to the user embedding.
Table \ref{tab:eval} presents our evaluation results.
$\Feat2Vec$ outperforms CBOW in the MPR metric.
Additionally, while CBOW predicts the actual director 1.26\% of the times, $\Feat2Vec$ does so 2.43\% of the time, a 92\% improvement in Top-1 Precision over CBOW.



\begin{table}[htb]
\footnotesize
\centering
\caption{Mean percentile rank}
\begin{tabular}{lcc}
\hline
Dataset & $\Feat2Vec$ & CBOW \\ \hline
IMDB & 19.36\% & 24.15\% \\
Educational & 25.2\% & 29.2\% \\
\hline
\end{tabular}
\label{tab:eval}
\end{table}



\subsubsection{Self-Supervised $\Feat2Vec$ Performance with Continuous Inputs}



We now focus on how well our estimated self-supervised $\Feat2Vec$ model performs when predicting a real-valued feature.  We expect this task to highlight $\Feat2Vec$'s advantage over token-based embedding learning algorithms, such as $\Word2Vec$. For IMDB ratings, we use a 3-layer DNN extraction function that will require embeddings of numerically similar ratings to be close, while $\Word2Vec$ will treat two numerically different ratings as completely different entities.
We evaluate the prediction of IMDB ratings by choosing the rating embedding most similar\footnote{As before, the metric is cosine similarity.} to the embedding of the movie's director, and compute the MSE of the predicted rating in the test dataset.
 $\Feat2Vec$ scores an MSE of 6.6, while  $\Word2Vec$ (CBOW)  scores 9.3.



We also use the Yelp dataset, predicting the most similar rating embedding to the review text embeddings produced by Feat2Vec, Word2Vec (CBOW), and Doc2Vec (DM).
Doc2Vec is only used for the Yelp dataset because there is no obvious ``base document'' in the IMDB dataset for Doc2Vec to learn on, while in the Yelp dataset the review text is a natural candidate.
For Word2Vec and Doc2Vec, the review text embedding is the average of word embeddings, analogous to the context vector used for learning in these algorithms. Figure \ref{fig:confmat} reports the confusion matrices for each model from this experiment.
Word2Vec is poor in its predictive power, predicting 5 stars for 97\% of the test sample.
Though Feat2Vec and Doc2Vec yield comparable MSE (2.94 vs. 2.92, respectively), Feat2Vec outperforms Doc2Vec by a substantial margin in classification error rate (55\% vs. 73\%) and mean absolute error (1.13 vs. 1.31).
In general, Doc2Vec is unable to identify low rating reviews: only 0.4\% of Doc2vec predictions are $\leq$ 2 stars, despite this comprising 20\% of the data. In contrast, Feat2Vec is more diverse in its predictions, and better able to identify extreme reviews.


 \begin{figure}[t]
 \centering
 \includegraphics[width=\columnwidth]{../paper/output/yelp/text_rating_confusionmat_numbers}
 \caption{Confusion Matrices of Self-Supervised Yelp Ratings Predictions}
 \label{fig:confmat}
 \end{figure}


\section{Conclusion}

Embeddings have proven useful in a wide variety of contexts, but they are typically built from datasets with a single feature type as in the case of Word2Vec, or tuned for a single prediction task as in the case of Factorization Machine.
We believe $\Feat2Vec$ is an important step towards general-purpose embedding methods.
It decouples feature extraction from prediction for datasets with multiple feature types,  it can be self-supervised, and its embeddings are easily interpretable.

In the supervised setting,  $\Feat2Vec$ outperforms an algorithm specifically designed for text---even when using the same feature extraction CNN.
In the self-supervised setting, $\Feat2Vec$ exploits the structure of a dataset to learn embeddings in a more sensible way than existing methods. This yields performance improvements in our ranking and prediction tasks.
To the extent of our knowledge, self-supervised $\Feat2Vec$  is the first method to calculate continuous representations of data with multi-modal feature types and no explicit labels.

Future work could study how to reduce the amount of  human knowledge our approach requires;
for example by automatically grouping features into entities, or by automatically choosing a feature extraction function.
These ideas can extend to our codebase that we make available
\footnote{The $\Feat2Vec$ model code is available  here: \url{https://github.com/CheggEng/Feat2Vec}. The experiments using non-proprietary data are available here: \url{http://web.stanford.edu/~larmona/feat2vec/}. 
}.
Though further experimentation is necessary, we believe that our results are an encouraging step forward towards general-purpose embedding models.


{
\footnotesize
\footnotesize
\bibliographystyle{aaai}
\bibliography{joseg}
}




\end{document}
